\begin{deluxetable}{cccccc}
\tablecaption{Comparison between HYDRAD and EBTEL with $c_{1,cond}=c_{1,eqm}=2$ and $c_{1,cond}=6. The first three columns show the full loop length, heating pulse duration, maximum heating rate. The last three columns show $n_{max}$ for the three models. Only $n_{max} is shown as $T_{max}$ is relatively insensitive to the value of $c_1$. The first four rows correspond to the varying pulse durations considered in this paper. The last four rows are the four cases shown in Table 2 of \citet{cargill_enthalpy-based_2012}.}
\tablehead{\colhead{$2L$} & \colhead{$\tau$} & \colhead{$H_0$} & \colhead{HYDRAD, $n_{max}$} & \colhead{EBTEL, $n_{max}$} & \colhead{EBTEL ($c_{1,cond}$), $n_{max}$}\\ \colhead{Mm} & \colhead{s} & \colhead{erg cm$^{-3}$ s$^{-1}$} & \colhead{$10^8$ cm$^{-3}$} & \colhead{$10^8$ cm$^{-3}$} & \colhead{$10^8$ cm$^{-3}$}}
\startdata
80 & 20 & 1.0000 & 30.7 & 44.5 & 40.0 \\
80 & 40 & 0.5000 & 32.7 & 44.4 & 39.9 \\
80 & 200 & 0.1000 & 37.6 & 44.2 & 39.6 \\
80 & 500 & 0.0400 & 37.7 & 44.1 & 39.3 \\
150 & 500 & 0.0015 & 3.7 & 3.8 & 3.4 \\
50 & 200 & 0.0100 & 10.7 & 11.3 & 10.1 \\
50 & 200 & 2.0000 & 339.0 & 391.8 & 351.0 \\
50 & 200 & 0.0100 & 15.5 & 16.3 & 14.3
\enddata
\label{tab:table_c1_compare}
\end{deluxetable}
